\documentclass[17pt, t, lualatex]{beamer}



\title{Actividades, Grupos, Colectivos y Semilleros\footnote{Para sacar el máximo provecho a la matrícula :D } }
\date{\today}
\institute[UJTL]{Universidad Jorge Tadeo Lozano}
\author{Ludwig Alvarado Becerra}

\usepackage{amsmath, amssymb, mathtools}
\usepackage{graphicx}
\usepackage[spanish]{babel}
\usepackage{biblatex}
\usepackage{hyperref}
\usepackage{xurl}
\usepackage{svg}
\usepackage{listings}
\usepackage[scale=7]{ccicons}



\addbibresource{sample.bib}


\definecolor{white}{HTML}{FFFFFF}
\definecolor{sand}{HTML}{EBE5E0}
\definecolor{KTHblue}{HTML}{004791}
\definecolor{skyblue}{HTML}{6298D2}
\definecolor{navy}{HTML}{000061}
\definecolor{lightblue}{HTML}{DEF0FF}
\definecolor{digitalblue}{HTML}{0029ED}


\lstset{
  language=bash,                       % Set the language to bash
  backgroundcolor=\color{white},       % Background color for the code block
  basicstyle=\ttfamily\normalsize,    % Use typewriter font and small size
  keywordstyle=\color{KTHblue},           % Style for keywords (e.g. if, for, etc.)
  commentstyle=\color{sand},          % Style for comments
  stringstyle=\color{skyblue},             % Style for strings
  showstringspaces=false,              % Don't underline spaces in strings
  breaklines=true,                     % Automatically break long lines
  frame=single,                        % Add a frame around the code
  captionpos=b,                        % Position of the caption (bottom)
  numbers=left,                        % Line numbers on the left
  numberstyle=\tiny\color{gray},       % Style for line numbers
  stepnumber=1,                        % Display every line number
  numbersep=5pt,                       % Space between line numbers and code
  morekeywords={bat, wc, tail, cut, head, >},
}



% Probably load as late as possible
% Other options are
% - engine=pdflatex to compile in pdfLaTeX (with different fonts),
% - mathshape=rm to use serif font for math,
% - mathsahpe=custom to not set any math font (so that you can define your own math fonts)
\usetheme[engine=lualatex, mathshape=sf, fontdir=kthpq-files/fonts/Figtree/]{kthpq}
\setmonofont{Bitstream Vera Sans Mono}[Scale=.9]

% Custom colors (see beamercolorthemecustom.sty for more details)
% \usecolortheme{custom}

% Modify the headline template: KTH-full, KTH-section-only, or KTH-frametitle-only.
% \setbeamertemplate{headline}[KTH-full]

% Custom footline
% \setfootline{left}{center}{right}



\begin{document}

\inserttitlepage

\section{Índice}

\insertsectionpage

\begin{frame}
  \frametitle{Índice}
  \tableofcontents[hideallsubsections]
\end{frame}

\AtBeginSection[]{
  \begin{frame}
    \frametitle{Índice}
    \tableofcontents[currentsection, hideothersubsections]
  \end{frame}
}

\section{Deportes}

\begin{frame}
  \frametitle{Deportes}
  \begin{figure}
    \centering
    \includegraphics[height=0.9\textheight]{img/Deportes.png}
  \end{figure}

\end{frame}

\begin{frame}
  \frametitle{Deportes}
  \begin{columns}
    \begin{column}{.5\textwidth}
      \begin{itemize}
        \item Variedad de deportes.
        \item Estudiantes de todos los semestres y de todas las carreras.
        \item Más información en: \url{https://www.utadeo.edu.co/es/micrositio/deportes}
      \end{itemize}
    \end{column}

    \begin{column}{.5\textwidth}
      \begin{figure}
        \centering
        \includegraphics[width=0.8\textwidth]{img/TadeoOlimpiadas.jpg}
        \caption{Inauguración de las TadeOlimpiadas\cite{deportes2024}}
      \end{figure}
    \end{column}
  \end{columns}

\end{frame}


\section{Colectivo de Software Libre, GNUTADEO}

\insertsectionpage



\begin{frame}
  \frametitle{Colectivo de Software Libre, GNUTADEO}
  \begin{figure}
    \centering
    \includegraphics[height=0.85\textheight]{img/FLISOL2025.jpg}
    \caption{Participación FLISoL 2025}
  \end{figure}
\end{frame}

\begin{frame}
  \frametitle{Colectivo de Software Libre, GNUTADEO}
  \begin{figure}
    \centering
    \includegraphics[height=0.85\textheight]{img/FLISOL2024.jpg}
    \caption{Participación FLISoL 2024}
  \end{figure}
\end{frame}

\begin{frame}
  \frametitle{Colectivo de Software Libre, GNUTADEO}

  \begin{columns}
    \begin{column}{.5\textwidth}
      \begin{itemize}
        \item Promotores del conocimiento y acceso libre al mundo digital.
        \item Aportes a proyectos libres y de código abierto.
        \item Divulgadores del ecosistema GNU/Linux para técnicos y no-técnicos.
      \end{itemize}
    \end{column}

    \begin{column}{.5\textwidth}
      \begin{figure}
        \centering
        \includegraphics[width=1\textwidth]{img/Arch.png}
      \end{figure}
    \end{column}
  \end{columns}
\end{frame}


\section{Capítulo de YouthMappers, TadeoMappers}

\insertsectionpage

\begin{frame}
  \frametitle{OpenStreetMap y YouthMappers}
  \begin{columns}
    \begin{column}{.5\textwidth}
      \begin{itemize}
        \item Uno de los 4 grupos juveniles activos a nivel nacional de OpenStreetMap.
        \item Mapeo de árboles en instituciones de educación media.
        \item Participación y colaboración con diferentes universidades.
        \item Galardonado como \textit{YouthMappers Rising Stars Award}.
      \end{itemize}
    \end{column}

    \begin{column}{.5\textwidth}
      \begin{figure}
        \centering
        \includegraphics[width=\textwidth]{img/OSM.png}
      \end{figure}
    \end{column}
  \end{columns}
\end{frame}

\begin{frame}
  \frametitle{Reconocimiento}
  \begin{figure}
    \centering
    \includegraphics[height=0.9\textheight]{img/YM.jpg}
  \end{figure}
\end{frame}

\begin{frame}
  \frametitle{Evento Mapatón por Santa Fe}
  \begin{columns}
    \begin{column}{.3\textwidth}
      \begin{itemize}
        \item Colaboración con la Universidad Externado, Universidad Nacional, TomTom, IGAC, AC3 y Comunidad OSM Bogotá.
      \end{itemize}
    \end{column}

    \begin{column}{.7\textwidth}
      \begin{figure}
        \centering
        \includegraphics[width=\textwidth]{img/SantaFe.jpg}
      \end{figure}
    \end{column}
  \end{columns}
\end{frame}

\begin{frame}
  \frametitle{Evento Mapatón por Santa Fe}

  \begin{figure}
    \centering
    \includegraphics[angle = 270, width=0.3\textwidth]{img/Mapaton1.jpg}
  \end{figure}

\end{frame}

\begin{frame}
  \frametitle{Evento Mapatón por Santa Fe}

  \begin{figure}
    \centering
    \includegraphics[angle = 180, width=0.5\textwidth]{img/Mapaton2.jpg}
  \end{figure}

\end{frame}

\begin{frame}
  \frametitle{Mapeo de árboles por el ITI Francisco José de Caldas}
\begin{columns}
  \begin{column}{.5\textwidth}
  \begin{figure}
    \centering
    \includegraphics[height=0.8\textheight]{img/Mapeo1.jpg}
  \end{figure}
  \end{column}

  \begin{column}{.5\textwidth}
\begin{figure}
    \centering
    \includegraphics[height=0.8\textheight]{img/Mapeo2.jpg}
  \end{figure}
  \end{column}
\end{columns}


\end{frame}


\begin{frame}
  \frametitle{Resultados del Mapeo }
  \begin{figure}
    \centering
    \includegraphics[height=0.95\textheight]{img/ITI.png}
  \end{figure}
\end{frame}

\begin{frame}
  \frametitle{Mapeo de la Universidad en Google Maps }
  \begin{figure}
    \centering
    \includegraphics[height=0.95\textheight]{img/Google.png}
  \end{figure}
\end{frame}

\begin{frame}
  \frametitle{Mapeo de la Universidad en OpenStreetMap }

  \begin{figure}
    \centering
    \includegraphics[height=0.95\textheight]{img/UAntex.jpg}
  \end{figure}
\end{frame}

\begin{frame}
  \frametitle{Mapeo de la Universidad en OpenStreetMap }

  \begin{itemize}
    \item \url{https://www.openstreetmap.org/\#map=19/4.607024/-74.067595}
    \item \url{https://demo.f4map.com/\#lat=4.6070590&lon=-74.0675145&zoom=19}
  \end{itemize}

\end{frame}

\begin{frame}
  \frametitle{¡Únete!}
  \begin{columns}
    \begin{column}{.5\textwidth}
      \begin{itemize}
        \item Escribe a Ludwig.
        \item Sigue en \url{www.instagram.com/GNUTADEO}
      \end{itemize}
    \end{column}

    \begin{column}{.5\textwidth}
      \begin{figure}
        \centering
        \includegraphics[width=0.8\textwidth]{img/GNUTADEO.jpg}
      \end{figure}
    \end{column}
  \end{columns}
\end{frame}


\section{Semillero de Programación Competitiva, Segmentation Fault}

\insertsectionpage

\begin{frame}
  \frametitle{Participación en Maratones Nacionales}
  \begin{figure}
    \centering
    \includegraphics[height=0.95\textheight]{img/Maraton.jpg}
  \end{figure}
\end{frame}

\begin{frame}
  \frametitle{Beneficios}

  \begin{columns}
    \begin{column}{.5\textwidth}
      \begin{itemize}
        \item Resolución de problemas complejos.
        \item Competencias locales, nacionales, regionales e internacionales.
        \item Dentro del semillero se fomenta la comunidad y el trabajo en equipo.
        \item Conocimiento de estructuras de datos y algoritmos.
      \end{itemize}
    \end{column}

    \begin{column}{.5\textwidth}
      \begin{figure}
        \centering
        \includegraphics[width=\textwidth]{img/ICPC.jpg}
      \end{figure}
    \end{column}
  \end{columns}
\end{frame}

\begin{frame}
  \frametitle{Trabajo del Semillero}
  \begin{figure}
    \centering
    \includegraphics[height=0.95\textheight]{img/GitHub.png}
  \end{figure}
\end{frame}


\begin{frame}
  \frametitle{¿Interesado?}
  \begin{columns}
    \begin{column}{.5\textwidth}
      \begin{figure}
        \centering
        \includegraphics[width=0.8\textwidth]{img/YT.jpg}
        \caption{\url{www.youtube.com/@CPUtadeo}}
      \end{figure}
    \end{column}


    \begin{column}{.5\textwidth}
      \begin{figure}
        \centering
        \includegraphics[width=0.6\textwidth]{img/GH.png}
        \caption{\url{www.github.com/SegmentationFaultUtadeo/}}
      \end{figure}
    \end{column}

  \end{columns}
\end{frame}








\section{Referencias}
\begin{frame}
  \frametitle{Referencias}
  \printbibliography

\end{frame}

\insertendpage

\end{document}
